% Default: Pdflatex -> Bibtex -> Pdflatex -> Pdflatex
\documentclass{UniDraft}

% ============================================================
% Common settings 
% ============================================================

% ============================================================

% -------------------------------------------------------------
% use biblatex to manage references
% \usepackage[
%     doi=false,
%     isbn=false,
%     url=false,
%     eprint=false,
%     giveninits=true,
%     date=year,
%     sorting = none,
%     style = numeric-comp
% ]{biblatex}
% % biblatex_jiaxin: Jiaxin's customized bibliography file
% % biblatex: Other bibliography
% \addbibresource{biblatex_jiaxin.bib}
% \addbibresource{biblatex.bib}
% -------------------------------------------------------------

% -------------------------------------------------------------
\title{\bfseries \large Paper Title is Here}

% -------------------------------------------------------------
\author[1]{Jia-Xin Zhong}
\author[1]{B Author}
\author[2,3,*]{C Author}
\author[1,$\dagger$]{D Author}

\affil[1]{Key Laboratory of Modern Acoustics and Institute of Acoustics, Nanjing University, Nanjing 210093, China}
\affil[2]{Graduate Program in Acoustics, The Pennsylvania State University, University Park, PA 16802, USA}
\affil[3]{Affiliation 3}
% \affil[*]{stefano.longhi@polimi.it}
% \affil[$\dagger$]{yqj5201@psu.edu}

% \date{}
% Put email in date field
\date{\vspace{-1em}\small 
    % Email: 
    \textsuperscript{*}C.Author@psu.edu;
    \textsuperscript{$\dagger$}D.Author@nju.edu.cn
}

% Adding section symbol to section
% \usepackage{titlesec}
% \titleformat{\section}
% {\normalfont\Large\bfseries}{S\thesection}{1em}{}


% ============================================================
% 小节标题的风格选择,以下多种风格任选其一,取消注释即可
% ============================================================

% ------------------------------------------------------------
% 风格1:小节的编号为 S1、S2、S3 等的风格
% ------------------------------------------------------------
% \renewcommand*{\thesection}{S\arabic{section}}
% ------------------------------------------------------------

% ------------------------------------------------------------
% 风格2:小节的编号为 I.A、I.B、II.C 等风格
% ------------------------------------------------------------
% \renewcommand{\thesection}{\Roman{section}} 
% \renewcommand{\thesubsection}{\Roman{section}.\Alph{subsection}}
% ------------------------------------------------------------

% ------------------------------------------------------------
% 风格3
% - TOC 和正文中小节的编号为 Section I.、A. 等风格(适用丁鲲推荐的PRL)
% - \ref 引用时仍保留 I 和 A 等风格
% ------------------------------------------------------------
% \renewcommand{\thesection}{\Roman{section}}
% \renewcommand{\thesubsection}{\Alph{subsection}}
% ------------------------------------------------------------

% ------------------------------------------------------------
% 设置正文中标题的格式;需要 titlesec 宏包
% ------------------------------------------------------------ 
% \titleformat{\section} % 设置一级标题(Section)格式
%   {\normalfont\Large\bfseries} % 字体格式:正常字体、大号、加粗
%   {Section~\thesection.}       % 标签格式:显示为 "Section 1." (注意这里加了点)
%   {1em}                        % 【关键】标签与标题文字之间的水平间距 (1em 约等于一个字符宽度)
%   {}                           % 标题前的代码(通常为空)

% \titleformat{\subsection}      % 设置二级标题(Subsection)格式
%   {\normalfont\large\bfseries} % 字体格式:比一级小一点、加粗
%   {\thesubsection.}            % 标签格式:显示为 "1.1." (手动加了一个点)
%   {1em}                        % 标签与标题文字之间的水平间距【可适当调整】
%   {}
% ------------------------------------------------------------

% ------------------------------------------------------------
% 风格4:小节标题前不加编号,只显示标题文字
% ------------------------------------------------------------
\titleformat{\section}         % 设置一级标题(Section)格式
  {\normalfont\Large\bfseries} % 字体格式:正常字体、大号、加粗
  {}                           % 【关键】标签留空,即不显示 "Section 1"
  {0pt}                        % 【关键】标签与标题文字之间的水平间距设为 0
  {}                           % 标题前的代码

\titleformat{\subsection}      % 设置二级标题(Subsection)格式
  {\normalfont\large\bfseries} % 字体格式:比一级小一点、加粗
  {}                           % 【关键】标签留空
  {0pt}                        % 【关键】间距设为 0
  {}
\titlespacing*{\section}{0pt}{3.5ex plus 1ex minus .2ex}{2.3ex plus .2ex}
\titlespacing*{\subsection}{0pt}{3.25ex plus 1ex minus .2ex}{1.5ex plus .2ex}
% ------------------------------------------------------------

% ============================================================

% ------------------------------------------------------------
% Abstract
\usepackage{abstract}

\renewcommand{\absnamepos}{flushleft} % Left align the abstract name
% Set font size and bold for abstract title
\renewcommand{\abstractnamefont}{\bfseries\large}
\setlength{\absleftindent}{0pt} % No indentation on the left
\setlength{\absrightindent}{0pt} % No indentation on the right
% Set font size and no indentation for abstract text
\renewcommand{\abstracttextfont}{\small\setlength{\parindent}{0pt}}

% ------------------------------------------------------------

%%%%%%%%%%%%%%%%%%%%%%%%%%%%%%%%%%%%%%%%%%%%%%%%%%%%%%%%%%%%%%%%
\begin{document}

% Enable line numbers for review. Comment to disable.
\linenumbers

% 把标题加入PDF书签,但不在目录里加上标题
\pdfbookmark[1]{Title}{title} % [1] 是书签层级,'title' 是内部引用标签

\maketitle
% \addcontentsline{toc}{part}{Title}
% enable Page 1 of xx at the first page

% \addcontentsline{toc}{section}{Title}

% \addcontentsline{toc}{section}{Contents}
% \vspace{-2em}

\begin{abstract}
    \lipsum[1]
\end{abstract}

\thispagestyle{firststyle}

\newpage

\section{Introduction}
Test reference \cite{Assouar2018AcousticMetasurfaces}.
\lipsum[2]

\section{Results}

\subsection{Subsection 1}
This is an example of equation:
\begin{equation}
    p = \iint\limits_{S} 
    v_z(x,y) G(x,y,z) \dd S
\end{equation}

\lipsum[3]

\subsection{Subsection 2}
\lipsum[4]

\section{Conclusions}
\lipsum[1]



\section{Acknowledgment}
Y. J. thanks the support of startup funds from Penn State University and NSF xxxx.

\section{Author contributions}

J.-X. Z., S. L., and Y. J. conceived the project.
S. L. and J.-X. Z. performed theoretical analysis and numerical simulations.
J.-X. Z. designed and performed the experiments with assistance from J. W. K..
J.-X. Z., S. L., and Y. J. wrote the paper.
S. L. and Y. J. supervised the project.

\section{Competing interests}
The authors declare no competing interests.


\clearpage
\bibliographystyle{unsrt}
\bibliography{bibtex}
% \printbibliography
\addcontentsline{toc}{section}{References}

\clearpage 
% ============================================================
% Methods 
% ============================================================
\section{Methods}

\subsection{Method 1}
\lipsum[5]
\subsection{Method 2}
\lipsum[6]



% ============================================================  

% ============================================================
% Figures are placed at the end of the manuscript
% ============================================================
\clearpage
\section{Figures}
\vfill
\begin{figure}[!htbp]
    \centering
    \phantomsection 
    \includegraphics[width=0.6\textwidth]{example-image}
    \caption{
        \textbf{One sentence summary of figure.}
        \lipsum[10]
    }
    \label{fig:schematic}
    \addcontentsline{toc}{subsection}{Figure \thefigure}
\end{figure}
\vfill

\clearpage
\begin{figure}[p]
    \centering
    \phantomsection 
    \includegraphics[width=0.95\textwidth]{example-image} 
    \caption{
        \textbf{One sentence summary of figure.}
        \lipsum[2]
    }
    \label{fig:exp_setup}
    \addcontentsline{toc}{subsection}{Figure \thefigure}
\end{figure}


% ============================================================

\end{document}


